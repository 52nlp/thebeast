During the course of this research we have encountered numerous
challenges that were not apparent at the start.  These challenges
raise some interesting research questions and practical issues one
must consider when embarking on exact inference using ILP.  The first
issue is that the generation of the ILP programs can take as long as
solving the ILP.  This leads us to wonder if their may be a way to
provide tighter integration of program generation and solving.  Such
an integration would avoid the need to query the model for \emph{all}
possible model components the solver may require.  

The use of ILP in other NLP tasks has provided a principled and
declarative manner to incorporate global linguistic constraints on the
system output.  This work lays the foundations for using similar
global constraints for translation.  However, such constraints may
require that we consider a higher-level representation of the input
and output sentence than just the words alone.  To achieve this it is
desirable, in the future, to investigate how to formulate other models
of machine translation within the ILP framework.

Discuss:
\begin{itemize}
\item Generating the ILP program can take as long as solving.
\item Tighter integration of program generation and the ILP solver.
\item How to handle larger language models.  Left-to-right decoders
  can easily accommodate this, ILP not so.
\item Limitations of ILP wrt to language model.
\end{itemize}

Future work:
\begin{itemize}
\item Other MT models as ILP, will these be faster/slower etc?
\item Injection of linguistic knowledge through constraints.
\end{itemize}

%%% Local Variables: 
%%% mode: latex
%%% TeX-master: "ilp-mt"
%%% End: 
