
\global\long\def\source{\mathbf{e}}
\global\long\def\target{\mathbf{f}}
\global\long\def\align{\mathbf{a}}
\global\long\def\start{\text{START}}
\global\long\def\stop{\text{END}}
\global\long\def\null{\text{NULL}}
\global\long\def\sourceset{S}


Given a trained IBM model 4, and a French sentence $\target$ we need
to find the English sentence $\source$ and alignment $\align$ with
maximal $p\left(\align,\source|f\right)\backsimeq p\left(\source\right)\cdot p\left(\align,\target|\source\right)$.
\citet{germann01fast} showed that we can formulate this problem as a variant of the Travelling Salesman Problem (actually Knight showed that before) and presented an Integer Linear Programming~(ILP) formulation of this problem. 

In this section we will give a very high-level description of this ILP formulation.%
\footnote{Note that our formulation differs slightly because we use a first order modelling
language that imposed certain restrictions on the type of constraints
allowed.%
} 
 For brevity we refer the reader to the original work for details of the ILP formulation. 

In their ILP formulation a path through a set of English candidate tokens is represented through a set of binary variables that denote whether or not two tokens are directly connected through this path. Among constraints which guarantee that each French word has exactly one English word that generates it, the program also contains an exponential number of constraints that forbid each possible cycle the variables can represent. It is this set of constraints that renders decoding with ILP difficult. 

 



