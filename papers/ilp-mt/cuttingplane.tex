The Integer Linear Program we have described above has an exponential
number of (cycle) constraints. Hence, simply passing the ILP to an
off-the-shelf ILP solver is not practical for all but the smallest
sentences. For this reason the original {}``Optimal Decoding'' work
only considers sentences up to a length of 8? words. However, recent
work {[}riedel\&clarke{]} has shown that even exponentially large
MAP problems can efficiently solved using ILP solvers if a so-called
Cutting-Plane Algorithm is used. In the following we will present
this algorithm in a nutshell.
\begin{algorithm}
Cutting Plane algorithm for MT
\begin{enumerate}
\item Construct ILP $I$ without cycle constraints
\item \textbf{do}

\begin{enumerate}
\item solve $I$ and assign to $y$
\item find cycles in solution $y$
\item add corresponding cycle constraints to $I$
\end{enumerate}
\textbf{until} no more cycles can be found

\item return $y$
\end{enumerate}
\end{algorithm}
The Cutting Plane algorithm starts with a subset of the complete set
of constraints, namely all constraints but the (exponentially many)
cycle constraints. The corresponding ILP is solved by a standard ILP
solver, and the solution $y$ is inspected for cycles. If it contains
no cycles we are done (we have found the true optimum: the solution
with highest score that does not violate any constraints). If the
solution does contain cycles, the corresponding constraints are added
to the ILP which is in turn solved again. This process is continued
until no more cycles can be found. 

It is difficult to make claims about a guaranteed worst-case runtime
(or number of iterations) of this algorithm. However, if the linear
scoring function (in other words, the translation model and language
model parameters) already provides a preference for cycle-free solutions,
we can expect this algorithm to be efficient. For example, if we assume
that the translation/distortion model has a very strong preference
for monotonic solutions then clearly the highest scoring solution
is likely to be cycle-free.