
A formula is local if its groundings relate any number of observed ground atoms 
to exactly one hidden ground atom.  For example, a grounding of the local 
formula \[lemma(p,+l_1) \wedge lemma(a,+l_2) \Rightarrow hasRole(p,a)\]
connects a hidden \emph{hasRole/2} ground atom to two observed \emph{plemma/2} 
ground atoms. This formula can be interpreted as the feature for the predicate 
and argument lemmas in the argument identification stage of a pipeline SRL 
system.
Note that the ``+'' prefix indicates that there is a different weight for each 
possible pair of lemmas $(l_1,l_2)$.

We divide our local formulae into four sets, one for each hidden predicate.  For 
instance, the set for \emph{argument/1} only contains formulae in which the hidden 
predicate is \emph{argument/1}. 

The sets for \emph{argument/1} and \emph{sense/2} predicates have similar 
formulae since each predicate only involves one token at time: the SRL argument 
or the SRL predicate token. The formulae in these sets are defined using only 
\emph{token} or \emph{extended} observed predicates. 

There are two differences 
between the  \emph{argument/1} and \emph{sense/2} formulae.  First, the \emph{argument/1} formulae use 
the \emph{possibleArg/1} predicate as precondition, while the sense formulae are conditioned on 
\emph{predicate/1} predicate. For instance, consider the \emph{argument/1} 
formula based on word forms: \[word(a,+w) \land possibleArg(a) \Rightarrow 
argument(a),\] and the equivalent version for the \emph{sense/2} predicate: \[word(p,+w) 
\land predicate(p) \Rightarrow sense(p,+s).\] This means we only apply the 
\emph{argument/1} formulae if the token is a potential SRL argument, and 
the \emph{sense/2} formulae if the token is a SRL predicate. 

The second difference is the fact that for the \emph{sense/2} formulae we have different weights for each possible sense (as indicated by the $+s$ term in the second formula above), while for the \emph{argument/1} formulae this is not the case. This follows naturally from the fact that  \emph{argument/1}  do not explicitly consider senses. 

% For example, the word formulae presented above  formulae, the index for the
%irst formula is $(+w)$, which is the orthography, while in the second formula 
%has as indexes $(+w,+s)$, which is the orthography and the sense label.

What follows is a summary of the formulae in the \emph{argument/1} and 
\emph{sense/2} sets:
\begin{itemize}\addtolength{\itemsep}{-0.5\baselineskip}
    \item Word form of target token.
    \item Lemma of a token in a $-2 \cdots 2$ window around target token.
    \item POS tag of a token in a $-2 \cdots 2$ window around target token.
    \item Coarse POS tag of target token.
    \item Coarse POS of the two previous and two following tokens (conjoined).
    \item Dependency relation of the parent and children of target token.
    \item POS tag of children or parent of target token.
    \item POS tag of children or grandchildren and POS tag of target token.
    \item POS tag of parent or grandparent and POS tag of target token.
    \item Subcategorisation frame of target token.
\end{itemize}

The formulae in \emph{hasRole/2} and \emph{role/3} sets are conditioned on both 
the predicates \emph{possibleArg/1} and \emph{predicate/1} since both try 
to establish a relation between SRL predicates and arguments. For these predicate 
we use \emph{token}, \emph{extended} and \emph{path} predicates.

This is the summary for the formulae for the \emph{hasRole/2} set:
\begin{itemize}\addtolength{\itemsep}{-0.5\baselineskip}
    \item Lemma of predicate and argument.
    \item POS tags of predicate and argument.
    \item POS tag of predicate and POS tags of argument in a $-1,+1$ window.
    \item POS tag of predicate and lemma of argument.
    \item Lemma of predicate and POS tag of argument.
    \item Coarse POS tags for predicate and argument in a $-1,+1$ window.
    \item POS tag of argument and POS tag of children of argument.
    \item POS tag of argument and lemma  of children of argument.
    \item Dependency relation with parent for predicate and argument.
    \item Path and distance between predicate and argument.
    \item Path and frame of predicate and argument in combination with voice and 
        lemma.
\end{itemize}

This is the summary for the formulae for the \emph{role/3} set:
\begin{itemize}\addtolength{\itemsep}{-0.5\baselineskip}
    \item Lemma of predicate and argument.
    \item POS tag of predicate and lemma argument.
    \item POS tag of argument and POS tag of children of argument.
    \item POS tag of argument and lemma  of children of argument.
    \item Dependency relation with parent for predicate and argument.
    \item Path and frame and distance between predicate and argument.
    \item Path and frame of predicate and argument in combination with voice and 
        lemma.
    \item Path of predicate and argument in combination with lemma, POS tag, 
        coarse POS tag.
\end{itemize}

Note that these lists do not mention the feature information because 
this information was not available for every language. We therefore group the 
formulae which consider the \emph{feature/3} predicate into another a set we call 
\emph{feature} formulae. This is the summary of these formulae:
\begin{itemize}\addtolength{\itemsep}{-0.5\baselineskip}
    \item Feature-value pair of the token for \emph{sense/2}.
    \item Feature-value pair of the token for \emph{argument/1}.
    \item Feature-value pair for identical features of the predicate and argument 
        for \emph{hasRole/2}.
    \item Feature-value pair for identical features of the predicate and argument 
        for \emph{role/3}.
\end{itemize}

Additionally, we define a set of language specific formulae. They are aimed to 
capture the relations between argument and its siblings for the \emph{hasRole/2} 
and \emph{role/3} predicates.  In particular, these formulae were beneficial for 
the Japanese language.  This is a summary of such formulae which we called 
\emph{argument siblings}:
\begin{itemize}\addtolength{\itemsep}{-0.5\baselineskip}
    \item POS of argument and POS of argument sibling.
    \item POS of argument and lemma of argument sibling.
\end{itemize}

With these sets of formulae we can build specific MLNs for each language in 
the shared task. Table \ref{tbl:diff} shows the different configurations we use for the individual languages. We 
omit to mention the \emph{argument/1}, \emph{hasRole/2} and \emph{role/3} sets because they 
are present for all languages. 

% SR: This seems redundant given what was previously written 
% The \emph{feature} set corresponds to the FEAT 
% column provided in the corpus.  The presence of this set is determined by the 
% availability of this information in the corpus.  The presence of the 
% \emph{sense/2} set is determined by the labelling of senses in the corpora.  
% Finally, the formulae for the \emph{argument siblings} was implemented for the 
% Japanese during development. 

\begin{table}
\begin{center}
\small
\begin{tabular}{|l|c|c|c|}\hline
    Set         & Feature   & \emph{sense/2}  & Argument \\
                &            &        & siblings  \\\hline\hline
Catalan         &   Yes      &  Yes   &  No  \\
Chinese         &   No       &  Yes   &  No  \\
Czech           &   Yes      &  No    &  No  \\
English         &   No       &  Yes   &  No  \\
German          &   Yes      &  Yes   &  No  \\
Japanese        &   Yes      &  No    &  Yes \\
Spanish         &   Yes      &  Yes   &  No  \\
\hline
\end{tabular}
\caption{Difference among the formulae between the languages.}
\label{tbl:diff}
\normalsize
\end{center}
\end{table}

A more detailed description of the formulae can be found in our MLN model files.\footnote{\url{http://thebeast.googlecode.com/svn/mlns/conll09}} They can be 
used both as a reference and as input to our Markov Logic Engine,\footnote{\url{http://thebeast.googlecode.com}} and thus allow the reader 
to easily reproduce our results.


