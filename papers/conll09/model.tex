
In order to model the SRL task in the ML framework, we propose four hidden 
predicates. Consider the example of the previous section:
\begin{description}
    \item [\emph{argument/1}] indicates the phrase for which its head is a 
        specific position is an SRL argument. In our example     
        \emph{argument(2)} signals that the phrase for which the word in 
        position $2$ is its head is an argument (i.e., \emph{Ms. Haag}).
    \item [\emph{hasRole/2}] relates a SRL predicate to a SRL argument. For example, \emph{hasRole(3,2)} relates the
        predicate in position $3$ (i.e., \emph{play}) to the phrase which head 
        is in position $2$ (i.e., \emph{Ms. Haag}).
    \item [\emph{role/3}] identifies the role for a predicate-argument pair. For 
        example,
        \emph{role(3,2,ARG0)} denotes the role $ARG0$ for the SRL predicate in 
        the position 2 and the SRL argument in position 3.
    \item [\emph{sense/2}] denotes the sense of a predicate at a
        specific position. For example, \emph{sense(3,02)}
        signals that the predicate in position $3$ has the sense \emph{02}.
\end{description}

We also define three sets of observable predicates. The first set represents 
information about each token as provided in the shared task corpora for the 
closed track:  \emph{word} for the word form (e.g. \emph{word}(3,plays)); 
\emph{plemma/2} for the lemma; \emph{ppos/2} for the POS tag; \emph{feat/3} for 
each feature-value pair; \emph{dependency/3} for the head dependency and 
relation; \emph{predicate/1} for tokens that are predicates according to the 
``FILLPRED'' column. We will refer to these predicates as the \emph{token} 
predicates. 

The second set extends the information provided in the closed track corpus: 
\emph{cpos/2} is a coarse POS tag (first letter of actual POS tag); \emph{possibleArg/1} is true if the POS tag the token is a potential SRL argument  POS tag (e.g., PUNC is not); \emph{voice/2} denotes the voice for verbal tokens based on heuristics that use
syntactic information, or based on features in the FEAT column of the data. We will 
refer to these predicates as the \emph{extended} predicates.

Finally, the third set represents dependency information inspired by the features proposed by \citet{xue04calibrating}. There are two types of 
predicates in this set: \emph{paths} and \emph{frames}.  Paths capture the 
dependency path between two tokens, and frames the subcategorisation frame for a 
token or a pair of tokens. There are directed and 
undirected versions of paths, and labelled  (with dependency relations) and unlabelled versions of paths and frames. Finally, we have a frame predicate with the distance from the 
predicate to its head.  We will refer to the paths and most of the frames 
predicates as the \emph{path} predicates, while we will consider the 
\emph{frame} predicates for a unique token part \emph{token} predicates.

The ML predicates here presented are used within the formulae of our MLN.  We 
distinguish between two types of formula: local and global. 
