
In order to model the SRL task in the ML framework, we propose four hidden 
predicates, consider the example of the previous section:
\begin{description}
    \item [\emph{argument/1}] indicates the phrase for which its head is a 
        specific position is an SRL argument. For the example,     
        \emph{argument(2)} signals that the phrase for which the word in 
        position $2$ is its head is an argument (i.e., \emph{Ms. Haag}).
    \item [\emph{hasRole/2}] relates a SRL predicate to a SRL argument. For the 
        example, \emph{hasRole(3,2)} relates the
        predicate in position $3$ (i.e., \emph{play}) to the phrase which head 
        is in position $2$ (i.e., \emph{Ms. Haag}).
    \item [\emph{role/3}] besides to relating predicate and argument, it also
        identifies a role for the pair. For the example,
        \emph{role(3,2,ARG0)} signals the role for the predicate-argument role 
        presented above.
    \item [\emph{sense/2}] signals the sense identifier that a predicate in a
        specific position has. For the example, \emph{sense(3,02)}
        signals that the predicate in position $3$ has the sense \emph{02}.
\end{description}

We also define three sets of observable predicates. The first set are a direct 
map of the the information provided for each token in the closed Track of the 
Shared Task corpora:  \emph{word} for the orthography, \emph{plemma/2} for the 
lemma, \emph{ppos/2} for the POS tag, \emph{feature/3} for each feature-value 
pair, \emph{dependency/3} for the head dependency and relation, and 
\emph{predicate/1} if the token is a predicate. We will refer to these 
predicates as the \emph{token} predicates.

The second set extends the information provided by the columns of the corpus: 
\emph{cpos/2} is a coarse pos task, \emph{possibleArg/1} is true if based 
on heuristics on the POS tag the token is a potential SRL argument and 
\emph{voice/2} signals the voice for verbal tokens based on heuristics of the 
syntactic information or the features of the column FEAT of the corpora. We will 
refer to these predicates as the \emph{extended} predicates.

Finally the third set represent dependency information. These predicates are 
based on the features proposed by \citet{xue04calibrating}. We have two types of 
predicates for this set: \emph{paths} and \emph{frames}.  Paths capture the 
dependency path between two tokens, and frame the subcategorisation frame for a 
unique token or a part of tokens. For the path predicates we have a directed and 
undirected version. For both path and frame we have labelled and unlabelled 
versions.  Finally, we have a frame predicate with the distance from the 
predicate to its head.  We will refer to the paths and most of the frames 
predicates as the \emph{path} predicates, while we will consider the 
\emph{frame} predicates for a unique token part \emph{token} predicates.

We use the previous ML predicates to define the formulae of our MLN.  We 
distinguish between two types of formula: local and global. 
