
A formula is local if its groundings relate any number of observed ground atoms 
to exactly one hidden ground atom.  For example, a grounding of the local 
formula \[lemma(p,+l_1) \wedge lemma(a,+l_2) \Rightarrow hasRole(p,a)\]
connects a hidden \emph{hasRole/2} ground atom to two observed \emph{plemma/2} 
ground atoms. This formula can be interpreted as the feature for the predicate 
and argument lemmas in the argument identification stage of a pipeline SRL 
system.
Note that the ``+'' prefix indicates that there is a different weight for each 
possible pair of lemmas $(l_1,l_2)$.

We divide our local formulae into four sets, one for each hidden predicate.  For 
instance, the set for \emph{argument/1} only contains formulae in which the hidden 
predicate is \emph{argument/1}. 

The sets for \emph{argument/1} and \emph{sense/2} predicates have similar 
formulae since each predicate only involves one token at time: the SRL argument 
or the SRL predicate token. The formulae in these sets are defined using only 
\emph{token} or \emph{extended} observed predicates. 

There are two differences 
between the  \emph{argument/1} and \emph{sense/2} formulae.  First, the \emph{argument/1} formulae use 
the \emph{possibleArg/1} predicate as precondition, while the sense formulae are conditioned on 
\emph{predicate/1} predicate. For instance, consider the \emph{argument/1} 
formula based on word forms: \[word(a,+w) \land possibleArg(a) \Rightarrow 
argument(a),\] and the equivalent version for the \emph{sense/2} predicate: \[word(p,+w) 
\land predicate(p) \Rightarrow sense(p,+s).\] This means we only apply the 
\emph{argument/1} formulae if the token is a potential SRL argument, and 
the \emph{sense/2} formulae if the token is a SRL predicate. 

The second difference is the fact that for the \emph{sense/2} formulae we have different weights for each possible sense (as indicated by the $+s$ term in the second formula above), while for the \emph{argument/1} formulae this is not the case. This follows naturally from the fact that  \emph{argument/1}  do not explicitly consider senses. 

% For example, the word formulae presented above  formulae, the index for the
%irst formula is $(+w)$, which is the orthography, while in the second formula 
%has as indexes $(+w,+s)$, which is the orthography and the sense label.

%What follows is a summary of the formulae in the \emph{sense/2} sets (we omit 
%the \emph{predicate/1} precondition on the formulae, this is equivalent to 
%conjoin the \emph{predicate(t)} in the formulae):
Table \ref{tbl:f1} presents the templates of the local formuale for 
\emph{isArgument/1} and \emph{sense/2}. Templates are a way to briefly describe 
the FOL formulae of a ML. The template column shows the main body of the 
formulae, the parameters column shows the values that the parameters can take.For 
example, in the second row, the template contains the parameters $P$ and $i$ and its possible values. In total, this row represents $10$ 
formulae. For instance, one of these is: \[ plemma(p+1,+l) \], this is replaceing $plemma$ for P and $1$ for i. 
Additionally, these templates omit the precondition \emph{predicate(p)} which ensures the formulae is only apply to SRL predicates. This means the template should be read as: \[ predicate(p) \land  plemma(p+1,+l) \]
Finally, the last two columns indicate if there is a version of formula for the 
$isArgument/1$ (I) and/or $sense/2$ (S). In the case of our example, this means there are two formulae:
\[ predicate(p) \land plemma(p+1,+l) \land isArgument(p) \] and \[predicate(p) \land plemma(p+1,+l) \land sense(p,+s)\].

\begin{table*}
\centering
\begin{tabular}{|p{8cm}|c|c|c|}\hline
   Template       & Parameters & I & S \\\hline
   $ word(p,+w)$  &            & X & X \\
   $ P(p+i,+l)$   & $P \in \{plemma, ppos\}$, $ i \in \{-2..2\}$ & X & X \\
   $ cpos(p+1,+cp1) \land cpos(p-1,+cp2)$ & & X & X\\
   $ cpos(p+1,+cp1) \land cpos(p-1,+cp2) \land cpos(p+2,+cp3) \land cpos(p-2,+cp4)$ & & X & X\\
   $ dep(p,\_,+d)$ & & X & X\\
   $ dep(\_,p,+d)$ & & X & X\\
   $ ppos(j,+pj)  \land dep(p,j,+d)$ & & X & X \\
   $ ppos(j,+pj)  \land ppos(i,+pa) \land dep(p,j,+d)$ & & X & X \\
   $ ppos(i,+pj)  \land ppos(i,+pa) \land dep(i,j,\_) \land dep(j,p,+d)$ & & X & X \\
   $ plemma(j,+l) \land dep(j,j,+d)$ & & X & X\\
   $ frame(p,+f)$ & & X & X\\
                  & &   & X \\
\hline
\end{tabular}
\caption{Templates of the local formulae for \emph{isArgument/1} and 
\emph{sense/2}. I: head of clause is $isArgument(a)$, S: head of clause is 
$sense(p,+s)$}
\label{tbl:f1}
\end{table*}

The formulae in \emph{hasRole/2} and \emph{role/3} sets are conditioned on both 
the predicates \emph{possibleArg/1} and \emph{predicate/1} since both try 
to establish a relation between SRL predicates and arguments. For these predicate 
we use \emph{token}, \emph{extended} and \emph{path} predicates.

Table \ref{tbl:f2} shows a summary for the local formuale for \emph{hasRole/2} and \emph{role/3} predicates. This table follows similar rules of expansion than the described for the templates in \emph{tbl:f1}. The only difference is that these templetates are precondition by \[ predicate(p) \land possibleArg(a) \]. 
\begin{table*}
\centering
\begin{tabular}{|p{8cm}|p{4cm}|c|c|}\hline
   Template               & Parameters & H & R \\\hline
   $                 $    &                        & X & X \\
   $ P(p,+w)         $    & $P \in \{plemma,ppos\}$ & X & X \\
   $ P(p,+w) \land +d$    & $P \in \{plemma,ppos\}$, $d \in \{0,1,2,3,4,5,10\}$ & X &   \\
   $ plemma (p,+w_1) \land ppos(a,+w_2) $    & & X &  \\
   $ ppos(p,+w_1) \land plemma(a,+w_2) $    & & X &  \\
   $ plemma(p,+w_1) \land plemma(a,+w_2) \land +d $ & $d \in \{0,1,2,3,4,5,10\}$ & X & X \\
   $ ppos(p,+w_1) \land ppos(a,+w_2) \land +d $ & $d \in \{0,1,2,3,4,5,10\}$ & X &   \\
   $ ppos(p,+w_1) \land ppos(a+i,+w_2) \land +d $ & $i \in \{-1,0,1\}$, $d \in \{0,1,2,3,4,5,10\}$ & X &   \\
   $ lemma(p,+w) \land +d $ & $d \in \{0,1,2,3,4,5,10\}$ &  & X  \\
   $ lemma(a,+w) \land voice(p,+w_2) \land +d $ & $d \in \{0,1,2,3,4,5,10\}$ &  & X  \\
   $ cpos(p,+cp1) \land cpos(p+i,+cp2) \land cpos(a,+cp3) \land cpos(a+j,+cp4)$ & $i,j \in \{-1,1\}^2$ & X & X\\
   $ ppos(p,+w_1) \land ppos(a,IN) \land dep(a,m,\_) \land P(m,+w_2) $ & $P \in \{ppos, plemma \}$ & X &  X \\
   $ plemma(p,+w_1) \land ppos(a,IN) \land dep(a,m,\_) \land pposm,+w_2) $ & & X &  X \\
   $ P(p,a,+w)         $    & $P \in \{frame, unlabelFrame, path\}$ & X & X \\
   $ P(p,a,+w_1) \land plemma(p,+w_2) $ & $P \in \{frame,pathFrame\}$             & X & X \\
   $ P(p,a,+w_1) \land plemma(p,+w_2) \land plemma(a,+w_3) $    &   $P \in \{frame,pathFrame,path\}$   & X & X \\
   $ pathFrame(p,a,+w_1) \land lemma(p,+w_2) \land voice(p,+v) $ &            & X & X \\
   $ pathFrameDist(p,a,+d) $ &  $d \in \{0,1,2,3,4,5,10\}$           & X & X \\
   $ pathFrameDist(p,a,+d) \land voice(p,+w) $ &  $d \in \{0,1,2,3,4,5,10\}$           & X & X \\
   $ pathFrameDist(p,a,+d) \land plemma(p,+w) $ & $d \in \{0,1,2,3,4,5,10\}$           & X & X \\
   $ P(p,a,+d) \land plemma(a,+w) $ & $P \in \{pathFrameDist, path\}$, $d \in \{0,1,2,3,4,5,10\}$           & X & X \\
   $ P(p,a,+d) \land ppos(p,+w) $ & $P \in \{pathFrameDist, path\}$, $d \in \{0,1,2,3,4,5,10\}$           & X & X \\
   $ pathFrameDist(p,a,+d) \land ppos(p,+w_1) \land ppos(a,+w_2) $ & $d \in \{0,1,2,3,4,5,10\}$           & X & X \\
   $ patf(p,a,+d) \land plemma(p,+w_1) \land cpos(a,+w_2) $ & $d \in \{0,1,2,3,4,5,10\}$           & X & X \\
   $ dep(\_,a,+w) \land +d$    & $d \in \{0,1,2,3,4,5,10\}$ & X &   \\
   $ dep(\_,a,+w_1) \land voice(p,+w_2) \land +d$    & $d \in \{0,1,2,3,4,5,10\}$ & X &   \\
   $ dep(\_,a,+w_1) \land dep(\_,p,+w_2) \land +d$    & $d \in \{0,1,2,3,4,5,10\}$ & X &   \\
\hline
\end{tabular}
\caption{Templates of the local formulae for \emph{hasRole/2} and 
\emph{role/3}. H: head of clause is $hasRole(p,a)$, R: head of clause is 
$role(p,a,+r)$}
\label{tbl:f2}
\end{table*}

Note that Tables \ref{tbl:f1,tbl:f2} do not mention the feature information because 
this information was not available for every language. We therefore group the 
formulae which consider the \emph{feature/3} predicate into another a set we call 
\emph{feature} formulae. This is the summary of these formulae:
\begin{tabular}{p{7.0cm}}
   $ feature(p,+f,+v) \land sense(p,+s)    $\\
   $ feature(p,+f,+v) \land isArgument(a)    $\\
   $ feature(p,+f,+v1) \land feature(p,f,+v2) \land hasRole(p,a)    $\\
   $ feature(p,+f,+v1) \land feature(p,f,+v2) \land role(p,a,+r)   $\\
\end{tabular}


Additionally, we define a set of language specific formulae. They are aimed to 
capture the relations between argument and its siblings for the \emph{hasRole/2} 
and \emph{role/3} predicates.  In particular, these formulae were beneficial for 
the Japanese language.  This is a summary of such formulae which we called 
\emph{argument siblings}:
\begin{tabular}{p{7.0cm}}
   $ dep(a,h,\_) \land dep(h,c,\_) \land ppos(a,+p1) \land ppos(c,+p2) \land hasRole(p,a)    $\\
   $ dep(a,h,\_) \land dep(h,c,\_) \land ppos(a,+p1) \land ppos(c,+p2) \land role(p,a,+r)    $\\
   $ dep(a,h,\_) \land dep(h,c,\_) \land plemma(a,+p1) \land ppos(c,+p2) \land hasRole(p,a)    $\\
   $ dep(a,h,\_) \land dep(h,c,\_) \land plemma(a,+p1) \land ppos(c,+p2) \land role(p,a,+r)    $\\
\end{tabular}

With these sets of formulae we can build specific MLNs for each language in 
the shared task. Table \ref{tbl:diff} shows the different configurations we use for the individual languages. We 
omit to mention the \emph{argument/1}, \emph{hasRole/2} and \emph{role/3} sets because they 
are present for all languages. 

% SR: This seems redundant given what was previously written 
% The \emph{feature} set corresponds to the FEAT 
% column provided in the corpus.  The presence of this set is determined by the 
% availability of this information in the corpus.  The presence of the 
% \emph{sense/2} set is determined by the labelling of senses in the corpora.  
% Finally, the formulae for the \emph{argument siblings} was implemented for the 
% Japanese during development. 

\begin{table}
\begin{center}
\small
\begin{tabular}{|l|c|c|c|}\hline
    Set         & Feature   & \emph{sense/2}  & Argument \\
                &            &        & siblings  \\\hline\hline
Catalan         &   Yes      &  Yes   &  No  \\
Chinese         &   No       &  Yes   &  No  \\
Czech           &   Yes      &  No    &  No  \\
English         &   No       &  Yes   &  No  \\
German          &   Yes      &  Yes   &  No  \\
Japanese        &   Yes      &  No    &  Yes \\
Spanish         &   Yes      &  Yes   &  No  \\
\hline
\end{tabular}
\caption{Difference among the formulae between the languages.}
\label{tbl:diff}
\normalsize
\end{center}
\end{table}

A more detailed description of the formulae can be found in our MLN model files.\footnote{\url{http://thebeast.googlecode.com/svn/mlns/conll09}} They can be 
used both as a reference and as input to our Markov Logic Engine,\footnote{\url{http://thebeast.googlecode.com}} and thus allow the reader 
to easily reproduce our results.



