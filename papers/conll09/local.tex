
A formula is local if its groundings relate any number of observed ground atoms 
to exactly one hidden ground atom.  For example, a grounding of the local 
formula \[lemma(p,+l_1) \wedge lemma(a,+l_2) \Rightarrow hasRole(p,a)\]
connects a hidden \emph{hasRole/2} ground atom to two observed \emph{plemma/2} 
ground atoms. This formula can be interpreted as the feature for the predicate 
and argument lemmas in the argument identification stage of a pipeline SRL 
system.
Note that the ``+'' prefix indicates that there is a different weight for each 
possible pair of lemmas $(l_1,l_2)$.

We divide our local formulae into four sets, one for each hidden predicate.  For 
instance, the set for \emph{argument/1} only contains formulae in which the hidden 
predicate is \emph{argument/1}. 

The sets for \emph{argument/1} and \emph{sense/2} predicates have similar 
formulae since each predicate only involves one token at time: the SRL argument 
or the SRL predicate token. The formulae in these sets are defined using only 
\emph{token} or \emph{extended} observed predicates. 

There are two differences 
between the  \emph{argument/1} and \emph{sense/2} formulae.  First, the \emph{argument/1} formulae use 
the \emph{possibleArg/1} predicate as precondition, while the sense formulae are conditioned on 
\emph{predicate/1} predicate. For instance, consider the \emph{argument/1} 
formula based on word forms: \[word(a,+w) \land possibleArg(a) \Rightarrow 
argument(a),\] and the equivalent version for the \emph{sense/2} predicate: \[word(p,+w) 
\land predicate(p) \Rightarrow sense(p,+s).\] This means we only apply the 
\emph{argument/1} formulae if the token is a potential SRL argument, and 
the \emph{sense/2} formulae if the token is a SRL predicate. 

The second difference is the fact that for the \emph{sense/2} formulae we have different weights for each possible sense (as indicated by the $+s$ term in the second formula above), while for the \emph{argument/1} formulae this is not the case. This follows naturally from the fact that  \emph{argument/1}  do not explicitly consider senses. 

% For example, the word formulae presented above  formulae, the index for the
%irst formula is $(+w)$, which is the orthography, while in the second formula 
%has as indexes $(+w,+s)$, which is the orthography and the sense label.

What follows is a summary of the formulae in the \emph{sense/2} sets (we omit the \emph{predicate/1} precondition on the formulae, this is equivalent to conjoin the \emph{predicate(t)} in the formulae):
\begin{tabular}{p{7.0cm}}
   $\cdot$ $ word(p,+wp) \land sense(p,+s)    $\\
   $\cdot$ $ plemma(p+i,+lp) \land sense(p,+s)$ for i in $[-2..2]$\\
   $\cdot$ $ ppos(p+i,+pp) \land sense(p,+s)$ for i in $[-2..2]$\\
   $\cdot$ $ cpos(p+1,+cp1) \land cpos(p-1,+cp2) \land sense(p,+s) $\\
   $\cdot$ $ cpos(p+1,+cp1) \land cpos(p-1,+cp2) \land cpos(p+2,+cp3) \land cpos(p-2,+cp4) \land sense(p,+s) $\\
   $\cdot$ $ dep(p,\_,+d) \land sense(p,+s) $\\
   $\cdot$ $ dep(\_,p,+d) \land sense(p,+s) $\\
   $\cdot$ $ ppos(j,+pj) \land dep(p,j,+d) \land sense(p,+s) $\\
   $\cdot$ $ ppos(j,+pj) \land ppos(i,+pa) \land dep(p,j,+d) \land sense(p,+s) $\\
   $\cdot$ $ ppos(i,+pj) \land ppos(i,+pa) \land dep(i,j,\_) \land dep(j,p,+d) \land sense(p,+s) $\\
   $\cdot$ $ plemma(j,+pj) \land dep(j,j,+d) \land sense(p,+s) $\\
   $\cdot$ $ frame(p,+f) \land sense(p,+s)$\\
\end{tabular}
Recall \emph{isArgument/1} formulae is similar.

The formulae in \emph{hasRole/2} and \emph{role/3} sets are conditioned on both 
the predicates \emph{possibleArg/1} and \emph{predicate/1} since both try 
to establish a relation between SRL predicates and arguments. For these predicate 
we use \emph{token}, \emph{extended} and \emph{path} predicates.

This is the summary for the formulae for the \emph{hasRole/2} set (we omit the \emph{predicate/1} and \emph{possibleArg} precondition, this is equivalent to conjoin $predicate(p) \land possibleArg(a)$ in the formulae):
\begin{tabular}{p{7.0cm}}
   $\cdot$ $ plemma(p,+lp) \land hasRole(p,a)  $\\
   $\cdot$ $ ppos(p,+pp) \land hasRole(p,a)  $\\
   $\bullet$ $ plemma(p,lp) \land hasRole(p,a)  $\\
   $\bullet$ $ ppos(p,pp) \land hasRole(p,a)  $\\
\end{tabular}

\begin{tabular}{p{7.0cm}}
   $\cdot$ $ ppos(p,+cp) \land hasRole(p,a)$\\
   $\cdot$ $ plemma(p,+l) \land hasRole(p,a)$\\
   $\cdot$ $ plemma(p,+lp) \land plemma(a,+la) \land hasRole(p,a)$\\
   $\bullet$ $ ppos(p,+pp) \land ppos(a+i,+pa) \land hasRole(p,a)$ for i in $[-1..1]$\\
   $\cdot$ $ plemma(p,+lp) \land ppos(a,+pa) \land hasRole(p,a)$\\
   $\bullet$ $ ppos(p,+pp) \land plemma(a,+la) \land hasRole(p,a)$\\
\end{tabular}

\begin{tabular}{p{7.0cm}}
   $\bullet$ $ cpos(p,+cp1) \land cpos(p-i,+cp2) \land cpos(a,+cp3) \land cpos(a+j,+cp4) \land hasRole(p,a)$ for (i,j) in $[(1,1),(-1,1),(1,-1),(1,1)]$\\
   $\cdot$ $ppos(p,pp) \land ppos(a,IN) \land dep(a,m,\_) \land ppos(m,+pm) \land hasRole(p,a)$\\
   $\cdot$ $ ppos(p,pp) \land ppos(a,IN) \land dep(a,m,\_) \land plemma(m,+lm) \land hasRole(p,a)$\\
   $\cdot$ $ plemma(p,+lp) \land ppos(a,IN) \land dep(a,m,\_) \land ppos(m,+pm) \land hasRole(p,a)$\\
\end{tabular}

\begin{tabular}{p{7.0cm}}
    $\cdot$ $frame(p,a,+f) \land hasRole(p,a)$\\
    $\cdot$ $unlabelFrame(p,a,+f) \land hasRole(p,a)$\\
    $\cdot$ $plemma(p,+lp) \land unlabelFrame(p,a,+f) \land hasRole(p,a)$\\
    $\cdot$ $plemma(p,+lp) \land plemma(a,+la) \land unlabelFrame(p,a) \land hasRole(p,a,+r)$\\
    $\cdot$ $ plemma(p,+lp) \land pathFrame(p,a,+f) \land hasRole(p,a)$\\
\end{tabular}

\begin{tabular}{p{7.0cm}}
    $\cdot$ $ plemma(p,+lp) \land plemma(a,+la) \land pathFrame(p,a,+f) \land 
    hasRole(p,a)$\\
    $\cdot$ $ plemma(p,+lp) \land voice(p,+v) \land pathFrame(p,a,+f) \land hasRole(p,a)$\\
    $\cdot$ $ plemma(p,+lp) \land plemma(a,+la) \land pathFrame(p,a,+f) \land 
    hasRole(p,a)$\\
    $\cdot$ $ pathFrameDistance(p,a,d) \land hasRole(p,a) \land d = 
    \left[0,1,2,3,4,5,10\right]$\\
\end{tabular}

\begin{tabular}{p{7.0cm}}
    $\diamond$ $ voice(p,+v) \land pathFrameDistance(p,a,d) \land hasRole(p,a)$\\
    $\diamond$ $ plemma(p,+lp) \land pathFrameDistance(p,a,d) \land hasRole(p,a)$\\
    $\diamond$ $ plemma(a,+la) \land pathFrameDistance(p,a,d) \land hasRole(p,a)$\\
    $\diamond$ $ ppos(p,+pp) \land pathFrameDistance(p,a,d) \land hasRole(p,a) $\\
    $\diamond$ $ ppos(p,+pp) \land ppos(a,+pa) \land pathFrameDistance(p,a,d) \land 
    hasRole(p,a)$\\
\end{tabular}

\begin{tabular}{p{7.0cm}}
    $\cdot$ $ dep(\_,a,+d) \land hasRole(p,a) \land | a-p | = 
    \left[0,1,2,3,4,5,10\right]$\\
    $\cdot$ $ dep(\_,a,+d) \land voice(p,+v) \land hasRole(p,a) \land | a-p | = 
    \left[0,1,2,3,4,5,10\right]$\\
    $\cdot$ $ dep(\_,a,+da) \land dep(\_,p,+dp) \land hasRole(p,a)$\\
    $\diamond$ $ dep(\_,a,+da) \land dep(\_,p,+dp) \land voice(p,+v) \land hasRole(p,a)$\\
\end{tabular}

\begin{tabular}{p{7.0cm}}
    $\cdot$ $ path(p,a,+p) \land hasRole(p,a)$\\
    $\cdot$ $ path(p,a,+p) \land ppos(a,+pa) \land hasRole(p,a)$\\
    $\cdot$ $ path(p,a,+p) \land plemma(a,+la) \land hasRole(p,a)$\\
    $\cdot$ $ path(p,a,+p) \land cpos(a,+pa) \land plemma(p,+lp) \land hasRole(p,a)$\\
    $\cdot$ $ path(p,a,+p) \land plemma(a,+la) \land plemma(p,+lp) \land hasRole(p,a)$\\
\end{tabular}

This is the summary for the formulae for the \emph{role/3} set:
\emph{Predicate:} Relates the information of the token of the predicate with the 
role (we omit the \emph{predicate/1} and \emph{possibleArg} precondition, this is equivalent to conjoin $predicate(p) \land possibleArg(a)$ in the formula)\\
\begin{tabular}{p{7.0cm}}
   $\cdot$ $ role(p,a,+r)$\\
   $\cdot$ $ ppos(p,+pp) \land ppos(a,+pa) \land role(p,a,+r)$\\
   $\bullet$ $ ppos(p,+cp) \land role(p,a,+r)$\\
   $\bullet$ $ plemma(p,+l) \land role(p,a,+r)$\\
   $\bullet$ $ plemma(p,+lp) \land plemma(a,+la) \land role(p,a,+r)$\\
   $\diamond$ $ plemma(a,+la) \land voice(p,+v) \land role(p,a,+r) \land p-a = \left[0,1,2,3,4,5,10\right]$\\
\end{tabular}

\begin{tabular}{p{7.0cm}}
   $\diamond$ $ cpos(p,+cp1) \land cpos(p+i,+cp2) \land cpos(a,+cp3) \land cpos(a+j,+cp4) 
    \land role(p,a,+r)$ for (i,j) in $[(1,1),(1,-1),(-1,1),(-1,-1)]$\\
   $\cdot$ $ ppos(p,pp) \land ppos(a,IN) \land dep(a,m,\_) \land ppos(m,+pm) \land 
    role(p,a,+r)$\\
   $\cdot$ $  ppos(p,pp) \land ppos(a,IN) \land dep(a,m,\_) \land plemma(m,+lm) \land 
    role(p,a,+r)$\\
   $\cdot$ $ plemma(p,+lp) \land ppos(a,IN) \land dep(a,m,\_) \land ppos(m,+pm) \land 
    role(p,a,+r)$\\
\end{tabular}

\begin{tabular}{p{7.0cm}}
   $\cdot$ $ frame(p,a,+f) \land role(p,a,+r)$\\
   $\cdot$ $ unlabelFrame(p,a,+f) \land role(p,a,+r)$\\
   $\cdot$ $ plemma(p,+lp) \land unlabelFrame(p,a,+f) \land role(p,a,+r)$\\
   $\cdot$ $ plemma(p,+lp) \land plemma(a,+la) \land unlabelFrame(p,a,+f) \land role(p,a,+r)$\\
   $\cdot$ $ plemma(p,+lp) \land pathFrame(p,a,+f) \land role(p,a,+r)$\\
   $\cdot$ $ plemma(p,+lp) \land plemma(a,+la) \land pathFrame(p,a,+f) \land role(p,a,+r)$\\
   $\cdot$ $ plemma(p,+lp) \land voice(p,+v) \land pathFrame(p,a,+f) \land role(p,a,+r)$\\
   $\cdot$ $ plemma(p,+lp) \land plemma(a,+la) \land pathFrame(p,a,+f) \land role(p,a,+r)$\\
   $\diamond$ $ pathFrameDistance(p,a,d) \land role(p,a,+r)$\\
   $\diamond$ $ voice(p,+v) \land pathFrameDistance(p,a,d) \land role(p,a,+r)$\\
   $\diamond$ $ plemma(p,+lp) \land pathFrameDistance(p,a,d) \land role(p,a,+r)$\\
   $\diamond$ $  plemma(a,+la) \land pathFrameDistance(p,a,d) \land role(p,a,+r)$\\
   $\diamond$ $ ppos(p,+pp) \land pathFrameDistance(p,a,d) \land role(p,a,+r)$\\
   $\diamond$ $ ppos(p,+pp) \land ppos(a,+pa) \land pathFrameDistance(p,a,d) \land role(p,a,+r)$\\
\end{tabular}

\begin{tabular}{p{7.0cm}}
  $\diamond$ $ dep(\_,a,+d) \land role(p,a,+r)$\\
  $\diamond$ $ dep(\_,a,+d) \land voice(p,+v) \land role(p,a,+r)$\\
  $\cdot$ $ dep(\_,a,+da) \land dep(\_,p,+dp) \land role(p,a,+r)$\\
  $\diamond$ $ dep(\_,a,+da) \land dep(\_,p,+dp) \land voice(p,+v) \land role(p,a,+r)$\\
  $\cdot$ $ path(p,a,+p) \land role(p,a,+r)$\\
  $\cdot$ $ path(p,a,+p) \land ppos(a,+pa) \land role(p,a,+r)$\\
  $\cdot$ $ path(p,a,+p) \land plemma(a,+la) \land role(p,a,+r)$\\
  $\cdot$ $ path(p,a,+p) \land cpos(a,+pa) \land plemma(p,+lp) \land role(p,a,+r)$\\
  $\cdot$ $ path(p,a,+p) \land plemma(a,+la) \land plemma(p,+lp) \land role(p,a,+r)$\\
\end{tabular}


Note that these lists do not mention the feature information because 
this information was not available for every language. We therefore group the 
formulae which consider the \emph{feature/3} predicate into another a set we call 
\emph{feature} formulae. This is the summary of these formulae:
\begin{tabular}{p{7.0cm}}
   $\cdot$ $ feature(p,+f,+v) \land sense(p,+s)    $\\
   $\cdot$ $ feature(p,+f,+v) \land isArgument(a)    $\\
   $\cdot$ $ feature(p,+f,+v1) \land feature(p,f,+v2) \land hasRole(p,a)    $\\
   $\cdot$ $ feature(p,+f,+v1) \land feature(p,f,+v2) \land role(p,a,+r)   $\\
\end{tabular}


Additionally, we define a set of language specific formulae. They are aimed to 
capture the relations between argument and its siblings for the \emph{hasRole/2} 
and \emph{role/3} predicates.  In particular, these formulae were beneficial for 
the Japanese language.  This is a summary of such formulae which we called 
\emph{argument siblings}:
\begin{tabular}{p{7.0cm}}
   $\cdot$ $ dep(a,h,\_) \land dep(h,c,\_) \land ppos(a,+p1) \land ppos(c,+p2) \land hasRole(p,a)    $\\
   $\cdot$ $ dep(a,h,\_) \land dep(h,c,\_) \land ppos(a,+p1) \land ppos(c,+p2) \land role(p,a,+r)    $\\
   $\cdot$ $ dep(a,h,\_) \land dep(h,c,\_) \land plemma(a,+p1) \land ppos(c,+p2) \land hasRole(p,a)    $\\
   $\cdot$ $ dep(a,h,\_) \land dep(h,c,\_) \land plemma(a,+p1) \land ppos(c,+p2) \land role(p,a,+r)    $\\
\end{tabular}


\begin{itemize}\addtolength{\itemsep}{-0.5\baselineskip}
    \item POS of argument and POS of argument sibling.
    \item POS of argument and lemma of argument sibling.
\end{itemize}

With these sets of formulae we can build specific MLNs for each language in 
the shared task. Table \ref{tbl:diff} shows the different configurations we use for the individual languages. We 
omit to mention the \emph{argument/1}, \emph{hasRole/2} and \emph{role/3} sets because they 
are present for all languages. 

% SR: This seems redundant given what was previously written 
% The \emph{feature} set corresponds to the FEAT 
% column provided in the corpus.  The presence of this set is determined by the 
% availability of this information in the corpus.  The presence of the 
% \emph{sense/2} set is determined by the labelling of senses in the corpora.  
% Finally, the formulae for the \emph{argument siblings} was implemented for the 
% Japanese during development. 

\begin{table}
\begin{center}
\small
\begin{tabular}{|l|c|c|c|}\hline
    Set         & Feature   & \emph{sense/2}  & Argument \\
                &            &        & siblings  \\\hline\hline
Catalan         &   Yes      &  Yes   &  No  \\
Chinese         &   No       &  Yes   &  No  \\
Czech           &   Yes      &  No    &  No  \\
English         &   No       &  Yes   &  No  \\
German          &   Yes      &  Yes   &  No  \\
Japanese        &   Yes      &  No    &  Yes \\
Spanish         &   Yes      &  Yes   &  No  \\
\hline
\end{tabular}
\caption{Difference among the formulae between the languages.}
\label{tbl:diff}
\normalsize
\end{center}
\end{table}

A more detailed description of the formulae can be found in our MLN model files.\footnote{\url{http://thebeast.googlecode.com/svn/mlns/conll09}} They can be 
used both as a reference and as input to our Markov Logic Engine,\footnote{\url{http://thebeast.googlecode.com}} and thus allow the reader 
to easily reproduce our results.



