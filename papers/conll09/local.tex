
A formula is local if its groundings relate any number of observed ground atoms 
to exactly one hidden ground atom.  For example, a grounding of the local 
formula \[plemma(p,+l_1) \wedge plemma(a,+l_2) \Rightarrow hasRole(p,a)\]
connects a hidden \emph{hasRole/2} ground atom to two observed \emph{lemma/2} 
ground atoms. This formula can be interpreted as the feature for the predicate 
and argument lemmas in the argument identification stage of a pipeline SRL 
system.
Note that the ``+'' prefix signals that there is a different weight for each 
possible pair of lemmas $(l_1,l_2)$.

We divide our local formulae into four sets, one for each hidden predicate.  For 
instance, the set for \emph{argument/1} only contains formulae which hidden 
predicate is \emph{argument/1}. 

The sets for \emph{argument/1} and \emph{sense/2} predicates have similar 
formulae since each predicate only involves one token at time: the SRL argument 
or the SRL predicate. The formulae in this sets are defined using only 
\emph{token} or \emph{extended} observed predicates. There are two differences 
among these two sets.  First, the \emph{argument/1} formulae are condition by
the \emph{possibleArg/1} predicate while sense are condition by 
\emph{predicate/1} predicate. For instance consider the \emph{argument/1} 
formula for the orthography: \[word(a,+w) \land possArgument(a) \land 
argument(a)\], and the contrast for the \emph{sense/2} predicate: \[word(p,+w) 
\land predicate(p) \land sense(p,+s)\]. This means we only apply the 
\emph{argument/1} formulae if the token is a potential SRL argument, and we 
only the \emph{sense/2} formulae if the token is a SRL predicate. The second 
difference is in the arity of the weights. \emph{sense/2} weights are larger by 
one than \emph{argument/2}. This is because their weights include the sense 
label as a part of its indexes. For the orthography formulae, the index for the
first formula is $(+w)$, which is the orthography, while in the second formula 
has as indexes $(+w,+s)$, which is the orthography and the sense label.

These is a summary of the formulae for the \emph{argument/1} and 
\emph{sense/2} sets:
\begin{itemize}\addtolength{\itemsep}{-0.5\baselineskip}
    \item Orthography of the token.
    \item Lemma of a token in a window of $-2 \cdots 2$ tokens.
    \item POS tag of a token in a window of $-2 \cdots 2$ tokens.
    \item Coarse POS tag for the token.
    \item Coarse POS of the two previous and two following tokens.
    \item Dependency relation of the parent and children of a token.
    \item POS tag of children or parent of a token.
    \item POS tag of children or grandchildren and POS tag of token.
    \item POS tag of parent or grandparent and POS tag of token.
    \item Subcategorisation frame of token.
\end{itemize}

The formulae in \emph{hasRole/2} and \emph{role/3} sets are condition both by 
the predicates \emph{possibleArg/1} and \emph{predicate/1} since both try 
to establish a relation between SRL predicate and argument. For these predicate 
we use \emph{toeken}, \emph{extended} and \emph{path} predicates.

This is the summary for the formulae for the \emph{hasRole/2} set:
\begin{itemize}\addtolength{\itemsep}{-0.5\baselineskip}
    \item Lemma of predicate and argument.
    \item POS tags of predicate and argument.
    \item POS tag of predicate and POS tags of argument in a window $-1,+1$.
    \item POS tag predicate and lemma argument.
    \item Lemma predicate and POS tag argument.
    \item Coarse POS tags for predicate and argument in a window $-1,+1$.
    \item POS tag argument and POS tag of children of argument.
    \item POS tag argument and lemma  of children of argument.
    \item Dependency relation with parent for predicate and argument.
    \item Path distance between predicate and argument.
    \item Path and frame of predicate and argument in combination with voice and 
        lemma.
\end{itemize}

This is the summary for the formulae for the \emph{role/3} set:
\begin{itemize}\addtolength{\itemsep}{-0.5\baselineskip}
    \item Lemma of predicate and argument.
    \item POS tag predicate and lemma argument.
    \item POS tag argument and POS tag of children of argument.
    \item POS tag argument and lemma  of children of argument.
    \item Dependency relation with parent for predicate and argument.
    \item Path and frame distance between predicate and argument.
    \item Path and frame of predicate and argument in combination with voice and 
        lemma.
    \item Path of predicate and argument in combination with lemma, POS tag, 
        coarse POS tag.
\end{itemize}

Note that these lists does not mention the feature information. This is because 
this information was not available for each language therefore we  group the 
formulae which contains this predicate into another a set we called 
\emph{feature} formulae. This is the summary of these formulae:
\begin{itemize}\addtolength{\itemsep}{-0.5\baselineskip}
    \item Feature-value pair of the token for \emph{sense/2}.
    \item Feature-value pair of the token for \emph{argument/1}.
    \item Feature-value pair for similar features of the predicate and argument 
        for \emph{hasRole/2}.
    \item Feature-value pair for similar features of the predicate and argument 
        for \emph{role/3}.
\end{itemize}

Additionally, we define a set of language specific formulae. They are aimed to 
capture the relations between argument and its siblings for the \emph{hasRole/2} 
and \emph{role/3} predicates.  In particular, these formulae was beneficial for 
the Japanese language.  This is a summary of such formulae which we called 
\emph{argument siblings}:
\begin{itemize}\addtolength{\itemsep}{-0.5\baselineskip}
    \item POS argument and POS of sibling.
    \item POS argument and lemma sibling.
\end{itemize}

With these set of formulae we can build a specific ML model for a language in 
the share task. Table \ref{tb:diff} shows the difference between languages. We 
omit the \emph{argument/1}, \emph{hasRole/2} and \emph{role/3} because they 
are present for all languages. The \emph{feature} set corresponds to the FEAT 
column provided in the corpus.  The presence of this set is determined by the 
availability of this information in the corpus.  The presence of the 
\emph{sense/2} set is determined by the labelling of senses in the corpora.  
Finally, the formulae for the \emph{argument siblings} was implemented for the 
Japanese during development. 

\begin{table}
\begin{center}
\small
\begin{tabular}{|l|c|c|c|}\hline
    Set             & Feature   & \emph{sense/2}  & Argument \\
                &            &        & siblings  \\\hline\hline
Catalan         &   Yes      &  Yes   &  No  \\
Chinese         &   No       &  Yes   &  No  \\
Czech           &   Yes      &  No    &  No  \\
English         &   No       &  Yes   &  No  \\
German          &   Yes      &  Yes   &  No  \\
Japanese        &   Yes      &  No    &  Yes \\
Spanish         &   Yes      &  Yes   &  No  \\
\hline
\end{tabular}
\caption{Different of formulae between the languages.}
\label{tbl:diff}
\normalsize
\end{center}
\end{table}

A more detail description of the formulae can be found in our MLN model files.  
\footnote{\url{http://thebeast.googlecode.com/svn/mlns/conll09}} They can be 
used both as a reference and as input to our Markov Logic 
Engine\footnote{\url{http://thebeast.googlecode.com}}, and thus allow the reader 
to easily reproduce our results.


