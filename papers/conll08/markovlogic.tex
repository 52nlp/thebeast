
%%% Local Variables: 
%%% mode: latex
%%% TeX-master: "paper"
%%% End: 

Markov Logic~\citep[ML,][]{richardson05markov} is a Statistical Relational Learning language based on First Order Logic and Markov Networks. It can be seen as a formalism that extends First Order Logic to allow formulae that can be violated
with some penalty. From an alternative point of view, it is an expressive
template language that uses First Order Logic formulae to instantiate
Markov Networks of repetitive structure. 

Let us describe Markov Logic by considering the predicate identification task. In Markov Logic we can model this task by first introducing a set of logical predicates\footnote{Notice that both the SRL task and in ML the notion of \emph{predicate} and \emph{argument} are used. Thus in the cases were is not obvious to which concept we refer we precede the prefix SRL or MLN, respectively.} such as \emph{isPredicate(Token)} or \emph{word(Token,Word)}. Then we specify a set of weighted first order formulae that define a distribution over sets of ground atoms of these predicates (or so-called \emph{possible worlds}). 

Ideally, the distribution we define through these weighted formulae assigns high probability to possible worlds where SRL predicates are correctly identified and a low probability to worlds where this is not the case. For example, a suitable set of weighted formulae would assign a high probability to the world\footnote{``Haag plays Elianti'' is a segment of a sentence in training corpus.}
\begin{eqnarray*}
 &\{ word\left(1,Haag\right),word(2,plays),\\
 & word(3,Elianti),isPredicate(2) \}& \end{eqnarray*}
and a low one to
\begin{eqnarray*}
& \{ word\left(1,Haag\right),word(2,plays),\\
 & word(3,Elianti),isPredicate(3) \} &\end{eqnarray*}
In Markov Logic a set $M=\left\{ \left(\phi,w_{\phi}\right)\right\} _{\phi}$ of weighted first order formulae is called a \emph{Markov Logic Network}~(MLN). It assigns a probability
\begin{equation}
\prob\left(\y\right)=\frac{1}{Z}\exp\left(
\sum_{\left(\phi,w\right)\in M} w
\sum_{\boldc\in C^{n_{\phi}}}f_{\boldc}^{\phi}\left(\y\right)
\right)
\label{eq:prob}
\end{equation}
to the possible world $\y$. Here $f_{\boldc}^{\phi}$ is a feature
function that returns 1.0 if in the possible world $\y$ the ground
formula we get by replacing the free variables in $\phi$ by the constants
in $\boldc$ is true and 0.0 otherwise. $C^{n_{\phi}}$ is the set
of all tuples of constants we can replace the free variables in $\phi$
with. $Z$ is a normalisation constant. Note that this distribution corresponds to a Markov Network where nodes represent ground atoms and factors represent ground formulae.

For example, if $M$ contains the formula $\phi$ \[
word\left(x,'take'\right)\Rightarrow isPredicate\left(x\right)\]
then its corresponding log-linear model has, among others, a feature 
$f_{t1}^{\phi}$ for which $x$ in $phi$ has been replaced by the constant $t_1$ and that returns 1.0 if \[
word\left(1,'take'\right)\Rightarrow isPredicate\left(1\right)\]
is true in $\y$ and 0.0 otherwise.

We will refer predicates such as \emph{word} as \emph{observed} because they are known in advance. In contrast, \emph{isPredicate} is \emph{hidden} because we need to infer it at test time.

\subsection{Learning}

An MLN we use to model the collective SRL task is presented in section \ref{sec:model}. We learn the weights associated this MLN using 1-best MIRA~\citep{crammer01ultraconservative} Online Learning method. %Note that we only consider formulae that appear more than once in the training set.

\subsection{Inference}

Assuming that we have an MLN with a set of suitable weights and a given sentence then we need to predict the choice of predicates, frame types, arguments and role labels with maximal a posteriori probability. To this end we apply a method that is both exact and efficient: Cutting Plane Inference~\citep[CPI,][]{riedel08improving} with Integer Linear Programming~(ILP) as base solver. We use it for inference at test time as well as during the MIRA online learning process.

Assuming that we have an MLN, a set of weights and a given sentence then we need to predict the choice of predicates, frame types, arguments and role labels with maximal a posteriori probability. To this end we apply a method that is both exact and efficient: Cutting Plane Inference~\citep[CPI,][]{riedel08improving} with Integer Linear Programming~(ILP) as base solver. We use it for inference at test time as well as during the MIRA online learning process.

