In this paper we have presented a Markov Logic Network that jointly
models all predicate identification, argument identification and
classification and sense disambiguation decisions for a sentence. We
have shown that this approach is competitive with state-of-the-art
results, using a relatively poor dependency parser. 

We demonstrated the
benefit of jointly predicting senses and semantic arguments when
compared to a pipeline system that first picks arguments and then
senses. We also showed that by modelling whether a token is an
argument of some predicate and jointly picking arguments for all
predicates of a sentence further improvements can be achieved.  

Finally, we demonstrated that our system is still efficient, despite
following a global approach. This efficiency was also shown to stem
from the first order inference method our Markov Logic engine
applies. 

We believe that a Markov Logic approach to Semantic Role Labelling may
also help us to answer interesting follow-up research questions: does it help to
enforce some type of ``subject raising'' constraints when looking at
multiple predicates at the same time? Can we integrate additional
stages of an actual NLP system (such as a dialogue or information
extraction system)? Can we integrate a dependency parser?     
